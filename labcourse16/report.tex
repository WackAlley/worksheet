\documentclass[%
aps,
onecolumn,
11pt,
tightenlines,
nofootinbib,
superscriptaddress,
floatfix,
prd,
]{revtex4-2}

\input{header}


\begin{document}

\title{\large Versuchsprotokoll - 16: Gammaquanten-Koinzidenzspektroskopie}

\author{Laura Gans-Bartl, Woo-Seok Yim}

\begin{abstract} 
	\vs{3mm}
	\centering
\begin{tcolorbox}
	\begin{center}		
		\textbf{Abstract}
	\end{center}\par
Mithilfe der Gammaquantenspektroskopie können koinzidente Gammaquanten aus Kernzerfällen selektiert werden. In diesem Versuch werden Kernzerfälle von $^{22}$Natrium und $^{60}$Cobalt untersucht. Als Detektoren werden Szintillationsdetektoren verwendet, bestehend aus Natriumiodid(NaI)-Kristallen mit Thallium(Tl)-Dotierung. Durch die Betrachtung der Winkelabhängigkeit der ausgesendeten Gammaquanten lassen sich Aussagen über die Art der Kernzerfälle treffen.\par 
Die Durchführung ist unterteilt in zwei Abschnitten: die Kalibration und die Koinzidenzspektroskopie von Natrium und Cobalt. Für die Kalibration werden die vollen Energiespektren der Quellen sowie die Zeitspektren bei zwei Verzögerungseinstellungen des Timing Single Channel Analysers (TSCA), 0$ns$ und 96$ns$, erfasst. Zu den Spektren werden passende Verteilungsfunktionen gefittet; in diesem Versuch verwenden wir zum numerischen Fit die Pythonbibliotheken \texttt{emcee} für multimodale Verteilungen und \texttt{scipy.optimize} für die unimodale Gaußverteilung.\par
Ziel der Koinzidenzspektroskopie ist es, die Kernzerfälle sowie die Winkelkorrelationen, die aus der Theorie bekannt sind, experimentell zu bestätigen. Aus diesem Versuch lernen wir auch über die Wichtigkeit der Zeitauflösung der gemessenen Signale am Gerät.
\end{tcolorbox}
\end{abstract}
%%%%%%%%%%%%%%%%%%%%%%%%%%%%%%%%%%%%%%%%%%%
\begin{center}
	\phantom{\fontsize{40}{40}\selectfont I}\\
	{\fontsize{20}{10}\selectfont Fortgeschrittenenpraktikum - \\[5mm]
	Versuch 16: $\gamma - \gamma$-Winkelkorrelationen und Koinzidenztechnik\\ [10mm]
	Gruppennummer: 2b\\[5mm]
	Versuchsassistent: Matthias Kleiner\\[5mm]
	Teilnehmer 1: Laura Gans-Bartl \\[5mm]
	Teilnehmer 2: Woo-Seok Yim \\[5mm]
	Durchgeführt am: 9. Dezember 2024 \\[5mm]
	Abgabedatum: 30. Dezember 2024 \\
		\hspace{100pt} 24. Januar 2025}
\end{center}
\newpage
\maketitle
\newpage
\tableofcontents
\newpage

%%%%%%%%%%%%%%%%%%%%%%%%%%%%%%%%%%%%%%%%%%%
%%%%%%%%%%%%%%%%%%%%%%%%%%%%%%%%%%%%%%%%%%%

\section{Einführung}
\label{sec:Einführung}
Kaskadenzerfälle $-$ das sind Zerfälle $A \to BC \to BXY$ mit einem temporären, metastabilen Zwischenzustand $-$ gehören zu den physikalischen Forschungsthemen mit weitreichenden Implikationen (Boyanovsky $\&$ Lello \cite{Boyanovsky_2014}). Die von Boyanovsky $\&$ Lello genannten Beispiele beschränken sich auf teilchenphysikalische Untersuchungen wie z.B. von B-Mesonzerfällen, die von der Large Hadron Collider beauty (LHCb) Collaboration durchgeführt wird \cite{Traczyk}. Es wird vermutet, dass Kaskadenzerfälle ein grundlegender Mechanismus hinter Paritätsverletzungen sein könnten, was wiederum Implikationen für das Standardmodell der Physik hat. \par
Ein Beispiel aus der Kernphysik ist die Detektion von Radionuklidzerfällen mittels Koinzidenzmessung von Sund\`en et \textit{al.}\cite{AnderssonSunden1901549}. Die Detektion wird für das Kontrollieren von Atomwaffentests verwendet und trägt somit eine besondere geopolitische Bedeutung. \par
Beide Beispiele haben die Gemeinsamkeit, dass präzise Messungen gewährleistet werden müssen: Im Falle von LHCb erfolgt die Detektion von Teilchen mit einem einarmigen Spektrometer und einem high-precision tracking system \cite{PhysRevD.84.092001}, während bei Koinzidenzmessungen eine hohe Zeitauflösung und das Minimieren von Hintergrundaktivitäten von höchster Priorität sind.\par
Das Ziel dieses Versuchs ist es, von der Theorie erwartete Koinzidenzen von Gammaquanten aus Kernzerfällen nachzuweisen und die Genauigkeit der Messmethode zu untersuchen.

%%%%%%%%%%%%%%%%%%%%%%%%%%%%%%%%%%%%%%%%%%%
%%%%%%%%%%%%%%%%%%%%%%%%%%%%%%%%%%%%%%%%%%%



\section{Theoretischer Hintergrund}
\label{sec:Theorie}

\subsection{Kernzerfälle}
\label{sec:Kernzerfälle}
Atome bestehen aus Nukleonen (Protonen und Neutronen) und Elektronen. Die Atomsorten $^{22}$Na und $^{60}$Co, die wir betrachten, werden auch als radioaktive Nuklide bezeichnet, und geben an, dass ihre Zusammensetzung instabil ist und sie zu kleineren Teilchen zerfallen. Dabei ist die Summe der Nukleonenzahl $A$ (auch Massezahl genannt) der entstandenen Teilchen gleich der des zerfallenen Teilchens. \par
Es gibt verschiedene Zerfallsarten; $\alpha, \beta^+, \beta^-$ und $\gamma-$Zerfall. Bei einem $\alpha-$Zerfall sendet das Atom ein $\alpha-$Teilchen aus, was einem $^4$He-Teilchen entspricht, und seine Massezahl verringert sich auf $A - 4$. Bei den $\beta-$Zerfällen wird entweder ein Neutron durch Aussenden eines Elektrons ($\beta^-$) und Antineutrinos ($\bar \nu$) in ein Proton umgewandelt, oder ein Proton in ein Neutron durch Aussenden eines Positrons ($\beta^+$) und Neutrinos ($\nu$). Bei den $\beta-$Zerfällen ändert sich die Massezahl nicht. \par
Wenn das entstandene Teilchen sich in einem angeregten Zustand befindet, geht es zum Grundzustand durch Emission von Photonen ($\gamma$). \par
Radionuklide unterscheiden sich in der Art und Reihenfolge dieser Zerfälle. Ihre Zerfälle kann man mit einem Zerfallsschema darstellen. Die Zerfallsschemata in Abbildung \ref{fig:decaysodium} und Abbildung \ref{fig:decaycobalt} sind folgendermaßen zu verstehen: rot mit Prozent angegeben sind die Wahrscheinlichkeiten der Zerfallsart; blau angegeben sind die $\gamma-$Zerfälle mit jeweils Energie in keV und Intensität/Wahrscheinlichkeit in Prozent, und in schwarz sind die Energieniveaus in keV. \vspace{15pt}\par
\textbf{Natrium Zerfallsschema} \par
Das Zerfallsschema von Natrium ist in Abbildung \ref{fig:decaysodium} zu sehen. $^{22}$Na zerfällt zunächst durch $\beta^+ -$Zerfall zu einem angeregten Neonatom, was wiederum durch das Aussenden eines 1274,5keV $\gamma-$Quants zum stabilen Neon zerfällt. Das besondere bei diesem Zerfall ist, dass das ausgesendete Positron mit einem Elektron annihilieren kann. Bei dieser Annihilation werden zwei entgegengesetzte $\gamma-$Quanten mit jeweils 511keV ausgesendet, welche zusammen der Ruheenergie des Elektron-Positron Paars entsprechen. Da diese Annihilation überwiegend aus einem ruhenden System hervorgeht \cite{manual1}, müssen die Ausrichtungen der beiden Gammaquanten gemäß der Impulserhaltung entgegengesetzt sein. \par
Das Zerfallsschema in Abbildung \ref{fig:decaysodium} zeigt zudem, dass zu 0.056$\%$ die Möglichkeit besteht, dass das Natriumatom direkt durch einen einzigen $\beta^+ -$Zerfall zu stabiles Neon zerfallen kann. Aufgrund der niedrigen Wahrscheinlichkeit ist dieser Zerfall im Versuch nicht zu erwarten.
\begin{figure}[ht]
	\includegraphics[scale=.4]{"images/Decay Sodium".png}
	\caption{Zerfallsschema $^{22}$Na, aus \url{https://www.nndc.bnl.gov/nudat3}\cite{BASUNIA201569}}
	\label{fig:decaysodium}
\end{figure}

\noindent \textbf{Cobalt Zerfallsschema}\\
\begin{figure}[ht]
	\includegraphics[width=\linewidth]{"images/Decay Cobalt".png}
	\caption{Zerfallsschema $^{60}$Co, aus \url{https://www.nndc.bnl.gov/nudat3}\cite{BROWNE20131849}}
	\label{fig:decaycobalt}
\end{figure}
Das Zerfallsschema von Coblat wird in Abbildung \ref{fig:decaycobalt} dargestellt. Cobalt zerfällt zunächst durch einen $\beta^- -$Zerfall zu einem angeregten Nickelatom. Dieses angeregte Nickelatom geht eine Reihe an $\gamma -$Zerfällen durch, was auch als Kaskadenzerfall bezeichnet wird. Wir beschränken uns hier ebenfalls auf die Zerfälle mit höchster Wahrscheinlichkeit; wie Abbildung \ref{fig:decaycobalt} zeigt, sind die Wahrscheinlichkeiten anderer Zerfälle vernachlässigbar klein. Die zu betrachtende Gammazerfallskaskade besteht aus einem 1173.2keV-Gammazerfall und einem 1332.5keV-Gammazerfall. 

\subsection{Winkelkorrelation}
\label{sec:Winkelkorrelation}
Während die Winkelabhängigkeit der Gammaquanten bei der Annihilationsstrahlung eindeutig ist (180°), besteht für die Winkelabhängigkeit der Gammaquanten des Kaskadenzerfalls eine winkelabhängige Intensitätsverteilung $W(\vartheta)$ \cite{manual1}:
\begin{align}
	W(\vartheta) &= 1 + A_2 P_2(\cos (\vartheta) + A_4 P_4(\cos (\vartheta)) \\
    P_2          &= \frac{1}{2}\left(3\cos^2(\vartheta) - 1 \right) \nonumber \\
    P_4          &= \frac{1}{8}\left(35\cos^4(\vartheta) - 30\cos^2(\vartheta) + 3 \right) \nonumber
\end{align}
welche aus der Multipolentwicklung der elektromagnetischen Strahlung hervorgeht. Hier stellen $P_k$ die Legendrepolynome $k-$ter Ordnung dar, während die Koeffizienten $A_k$ von den Spins der Energieniveaus und dem Mischungsverhältnis des Multipols abhängen. In unserem Fall der 4-2-0 Kaskade nehmen wir an, dass es keine Mischterme gibt, und die Koeffizienten direkt von der Tabelle der Winkelkorrelationskoeffizienten $F_k$ (Tabelle \ref{tab:angularcorrelationcoefficient}) übernommen werden.
\begin{table}[ht]
\begin{tabular}{llcrrrrr}
\toprule
 &  & $F_2(I,I',L=1, L'=1)$ & $F_2(1,2)$  & $F_2(2,2)$  & $F_4(2,2)$ & $F_4(2,3)$  & $F_4(2,4)$ \\
I & I’ &  &  &  &  &  &  \\
\midrule
\multirow[t]{3}{*}{1} & 0 & 0.7017 &  &  &  &  &  \\
 & 1 & -0.3536 & -1.0607 & -0.3536 &  &  &  \\
 & 2 & 0.0707 & 0.4743 & 0.3536 &  &  &  \\
\cline{1-8}
\multirow[t]{3}{*}{2} & 0 &  &  & -0.5976 & -1.0690 &  &  \\
 & 1 & -0.3354 & -0.9354 & -0.2988 & 0.7127 & 0.9960 & 0.0891 \\
 & 2 & -0.4183 & -0.6124 & 0.1281 & -0.3054 & -0.7986 & -0.1336 \\
 & 3 & 0.1195 & 0.6547 & 0.3415 & 0.0764 & 0.3260 & 0.0891 \\
 & 4 &  &  & -0.1707 & -0.0085 & -0.0627 & -0.0297 \\
\cline{1-8}
\multirow[t]{3}{*}{4} & 2 &  &  & -0.4477 & -0.3044 & 0.9004 & -0.0484 \\
 & 3 & 0.3134 & -0.9402 & -0.0448 & 0.6088 & -0.3035 & -0.1013 \\
 & 4 & -0.4387 & -0.3354 & 0.2646 & -0.4981 & -0.6139 & 0.0132 \\
\cline{1-8}
\bottomrule
\end{tabular}
\caption{Koeffizienten der Winkelverteilungsfunktion. Aus \cite{manual1}}
\label{tab:angularcorrelationcoefficient}
\end{table} \\
Für die Berechnung der Winkelverteilungsfunktion des 4-2-0 Kaskadenzerfalls folgt dann \cite{manual1}:
\begin{align}
	W(\vartheta) &= 1 + F_2(2,4,2,2)F_2(2,0,2,2)P_2(\vartheta) + F_4(2,4,2,2)F_4(2,0,2,2)P_4(\vartheta) \\
	W(\vartheta) &= 1 + 0.1020 P_2(\vartheta) + 0.009087 P_4(\vartheta). \nonumber
\end{align}
Die Winkelkorrelationen verschiedener Kaskadenzerfälle können dann gegen den Winkel aufgetragen werden, wie in Abbildung \ref{fig:angularcorrelation} dargestellt. Wir erkennen hierbei, dass es für den 4-2-0 Kaskadenzerfall einen "sweet-spot" Winkel gibt, bei der $W(\vartheta)=1$ beträgt. Dieser Winkel liegt ungefähr bei 55°, genau berechnet liegt er bei 53.28°. Dieser "sweet-spot" Winkel wird später für die Berechnung der Zählrate und der Quellstärke wichtig sein. Näheres dazu siehe \ref{sec:Koinzidenzspektroskopie}.
\begin{figure}[ht]
		\includegraphics[scale=1]{"plots/Winkelkorrelationen".pdf}
	\caption{Winkelkorrelationsfunktion für verschiedene Zerfallskaskaden. Hergestellt mit \texttt{matplotlib}}
	\label{fig:angularcorrelation}
\end{figure}

\subsection{Interaktionen und Detektion des Gammaquants}
\label{sec:Interaktionen}
In der Gammaspektroskopie betrachten wir die ausgesendeten Gammaquanten aus der Annihilationsstrahlung und aus dem $\gamma-$Zerfall von $^{22}$Na, sowie die Gammaquanten aus der Zerfallskaskade von $^{60}$Co. Im Allgemeinen interagieren diese Photonen sowohl als Teilchen (Bsp. Photoeffekt, Comptonstreuung) als auch als Welle (Bsp. Interferenz) mit der Natur; Für diesen Versuch ist der Gammaquant als interagierendes Teilchen relevant. Im Folgenden wird betrachtet, wie ein solcher Gammaquant mit einem Detektor auf atomarer Ebene interagiert. Wie im Abstract beschrieben, wird ein Szintillationsdetektor aus NaI-Kristallen mit Tl-Dotierung verwendet.\vspace{15pt}\par
\textbf{Photoeffekt und Tl-Dotierung} \par
Der Photoeffekt beschreibt, wie ein Elektron die Energie des Gammaquants absorbiert und in einem angeregten Zustand geht. Die Elektronen stammen von den NaI-Kristallen des Szintillationsdetektors. Dieses angeregte Elektron besetzt ein höheres Energieniveau und verlässt es wieder, entweder durch die Verteilung der Gammastrahlenergie über die umgebenen Elektronen, oder durch das Besetzen des leeren Energieniveaus durch ein anderes, höherenergetisches Sekundärelektron. Im letzteren Fall wird Röntgenstrahlung freigesetzt, was von der Energie her schwächer als die Gammastrahlung ist, trotzdem genug stark ist, um weitere Elektronen anzuregen. Somit entsteht ein Kreislauf, bis die letzten resultierenden Photonen nicht genug Energie für weitere Anregungen besitzen. Dies bedeutet, dass nur diese letzten Photonen den Detektor durchqueren, an der Photokathode ankommen und mit dem Photomultiplier (PMT) als Signal erkannt werden, während jegliche Photonen, die während des Kreislaufs ausgesendet werden, nur zum Kreislauf beitragen und nicht als Signal eingehen. Um die Zahl der signalgebenden Photonen und somit die Signalstärke zu erhöhen, werden die NaI-Kristalle mit Tl dotiert, was zu einer Einführung mehrerer zwischenliegender Energieniveaus führt. Die Elektronen, die sich zwischen diesen Niveaus bewegen, senden Photonen aus, die bereits zu wenig Energie für weitere Anregungen besitzen, und somit ohne den Durchgang des Kreislaufs direkt zur Photokathode ankommen. \vspace{15pt}\par
\textbf{Compton-Effekt}\par
Der Compton-Effekt beschreibt den Impulsstoß zwischen Photon und Elektron. Beim Stoß absorbiert das Elektron einen Teil der Photonenergie, wobei die Menge der Energie vom Winkel zwischen eingehenden Photon und Elektron abhängt:
\begin{align*}
	\lambda ' = \lambda + \frac{h}{m\cdot c} \left( 1-\cos (\vartheta) \right),
\end{align*}
wobei $\lambda$ die Wellenlänge des eingehenden Photons und $\lambda '$ die Energie des reflektierten Photons angibt. Die Reflektion ist somit bei 180° maximal, bei 0° minimal. Photonen, die durch die Comptonstreuung vom Detektor reflektiert werden, können bei günstig liegendem Winkel den \\ gegenüberliegenden Detektor erreichen und dort als Signal absorbiert werden. Wie wir bei den Energiespektren sehen werden, ist dieser Compton-Effekt für kleinere Peaks bei mittleren Energien verantwortlich.\vspace{15pt}\par
\textbf{Signalgebung}\par
Wie beim Photoeffekt besprochen, kommen die niederenergetischen Photonen an der Photokathode an. Diese passieren einen Sekundärelektronenvervielfacher und lösen freie Elektronen heraus. Die freien Elektronen werden durch eine Elektrode fokussiert und auf eine Dynode beschleunigt \cite{manual1}. Die beschleunigten Elektronen stoßen auf weitere Elektronen und lösen sie wiederum heraus, bis all diese Elektronen an der Anode ankommen und ein elektrisches Signal erzeugen. \cite{manual1}
\subsection{Koinzidenzspektroskopie}
\label{sec:Koinzidenzspektroskopie}
%%%%%%%%%%%%%%%%%%%%%%%%%%%%%%%%%%%%%%%%%%%
%%%%%%%%%%%%%%%%%%%%%%%%%%%%%%%%%%%%%%%%%%%
\noindent \textbf{Verteilungsspektren der Signale}\\
Eine dezidierte Betrachtung der statistischen Natur der Zählrate ist in \cite{gilmore2008practical} Kapitel 5, \textit{Statistics of Counting} zu finden, an der wir uns für die grobe Überlegung orientieren. Die für die Praxis wichtigen Formeln der Zählrate, Quellstärke und Winkelkorrelation sind von der Versuchsanleitung \cite{manual1} übernommen. Zweck der statistischen Betrachtung ist es, kurz zu beschreiben, welche Funktionen wir für die Parametriesierung der Energiespektra verwendet haben und warum. Auch könnte die statistische Betrachtung für das Parametrisieren der "Hintergrundsstrahlung" nützlich sein, die in diesem Versuch nicht auf theoretisch fester Basis durchgeführt wurde (Es wurde eine exponentielle oder lognormale Verteilung angenommen).\\
Die Anzahl an Signalen, die vom Detektor weiterverarbeitet werden, hängt von der Zahl der Photonen ab, die Zahl der Photonen wiederum von der Zerfallsrate $Q$ des Atoms:
\begin{align}
	Q &= \frac{dN}{dt} = \lambda N, 
\end{align}
mit der Anzahl an Atomen $N$ und der Zerfallskonstante $\lambda$. Das Auftreten eines Zerfalls ist diskret, vom anderen Zerfall unabhängig und frei von quantenmechanischen Interaktionen, weshalb wir gemäß Gilmore \textit{et al.} annehmen können, dass die Zerfallswahrscheinlichkeit binomial verteilt ist:
\begin{align}
	P_b(n) = \frac{N!}{(N - n)!n!}p^n(1-p)^{N-n},
\end{align}
welche die bekannte Formel der Binomialverteilung für $N$ Tupel(Gesamtzahl), $n$ Ereignissen \\(Zerfälle) und den Einzelfallwahrscheinlichkeiten $p$ ist. \\
Eine Näherung der Binomialverteilung für unbekannte $N$ ist die Poissonverteilung:
\begin{align}
	P_p(n) &= \frac{\mu^n(n)}{n!}e^{-\mu}
\end{align}
mit $\mu(n)$ dem Erwartungswert. Diese Verteilung geht für eine große Zahl an Ereignissen zur Normalverteilung über:
\begin{align}
	\label{eq:gaussian}
	P_N(x)\frac{1}{\sigma \sqrt{2\pi}} \exp \left[ -\frac{1}{2} \left( \frac{x-\mu}{\sigma} \right)^2 \right ]
\end{align}
mit $\mu$ dem Erwartungswert und $\sigma$ der Standardabweichung. \\
Für Verteilungen von positiven Zahlen kann statt der Normalverteilung die lognormale Verteilung verwendet werden:
\begin{align}
	\label{eq:lognormal}
	P_{logn}(x) &= \frac{1}{x \cdot \sigma \sqrt{2\pi}} \exp \left[ -\frac{1}{2} \left ( \frac{\ln x - \mu}{\sigma} \right )^2  \right].
\end{align}\par
\textbf{Zählrate, Quellstärke}\\
Die Zählrate eines Detektors setzt sich aus der Quellstärke/Zerfallsrate $Q$, dem Raumwinkel $w$ und der Ansprechempfindlichkeit $\epsilon$ des Detektors zusammen:
\begin{align}
	Z &= \epsilon w Q \\
	w &= \frac{f}{4 \pi r^2} \nonumber,
\end{align}
wobei $f$ die Detektorstirnfläche und $r$ der Abstand zwischen Detektor und Quelle ist. \\
Eine Koinzidenz besteht, wenn die Signale zweier Detektoren innerhalb eines Zeitfensters sind. Beim Beispiel der Vernichtungsstrahlung sollte der zeitliche Abstand der beiden Gammaquanten (bei gleichem Abstand zu ihrem jeweiligen Detektor) minimal sein und (theoretisch) stets kleiner als das Zeitfenster sein. Die Koinzidenzrate ist die Multiplikation der Zählrate eines Detektors mit der Auftrittswahrscheinlichkeit eines Signals beim zweiten Detektor. Diese Wahrscheinlichkeit beträgt bei einem zweiten Detektor 2 $\epsilon_2w_2W(\vartheta)$, wobei $W(\vartheta)$ die in \ref{sec:Winkelkorrelation} eingeführte Winkelkorrelationsfunktion ist. Insgesamt ergibt sich für die Koinzidenzrate $Z$:
\begin{align}
	Z_{Ke} &= Z_1 \cdot \epsilon_2w_2W(\vartheta) = \epsilon_1 w_1 Q \epsilon_2 w_2 W(\vartheta).
\end{align}
Hierbei handelt es sich um die "echte" Koinzidenzrate $Z_{Ke}$. Es besteht auch die Möglichkeit, dass sich zwei physikalisch unkorrelierte Signale per Zufall im selben Zeitfenster befinden und als Koinzidenz wahrgenommen werden. Diese "zufällige" Koinzidenzrate wird vor allem durch die endliche Koinzidenzauflösungszeit $\tau_K$ des signalverarbeitenden Geräts vergrößert. Die zufällige Koinzidenzrate besagt, wie oft ein Zweitsignal innerhalb der Pulslänge $\tau$ des ersten Signals auftritt. Wenn das Erstsignal mit einer Wahrscheinlichkeit von $Z_1 \tau$ auftritt, dann ergibt sich für die zufällige Koinzidenzrate $K_{Kz}$:
\begin{align}
	Z_{Kz} &= \tau Z_1 Z_2.
\end{align}
Das Verhältnis von echter und zufälliger Koinzidenzrate kann verwendet werden, um die Zerfallsrate und Winkelkorrelationsfunktion darzustellen:
\begin{align}
	\frac{Z_{Ke}}{Z_{Kz}} &= \frac{W(\vartheta)}{\tau Q} \\
	\frac{Z_{Ke}}{Z_1Z_2} &= \frac{W(\vartheta)}{Q}. \label{eq:qvalue}
\end{align}
Hier spielt auch der in \ref{sec:Winkelkorrelation} erwähnte "sweet-spot" Winkel die besondere Rolle, bei gemessenen $Z_{Ke}, Z_1$ und $Z_2$ die Berechnung der Zerfallsrate zu ermöglichen, wenn man beim Winkel 55° die Koinzidenzen misst.

\section{Signalverarbeitung}
\label{sec:Signalverarbeitung}
\begin{figure}[ht]
	\includegraphics[scale=0.14]{images/setting.jpg}
	\caption{Versuchsaufbau. Zu erkennen sind die zwei Detektoren und die signalverarbeitende Apparatur}
	\label{fig:setting}
\end{figure}
In Abbildung ref{fig:setting} zeigt, wie die Szintillationsdetektoren mit dem signalverarbeitendem Gerät verbunden sind. Die Apparatur besteht aus
\begin{itemize}
	\item dem \textit{Digital Data Link Amplifier} (DDL Amp) 
	\item dem \textit{Timing Single Channel Analyser} (TSCA), das das vom PMT angekommene Signal in ein starkes positives Rechtecksignal von 5V und ein schwaches negatives Rechtecksignal von -0.5V umwandelt
	\item dem \textit{Time to Pulse Height Converter} (TPHC)
	\item dem \textit{Analog to Digital Converter} (ADC), dessen \textit{Gate} die Funktion eines input gates spielt, d.h. ein einkommendes Signal anzunehmen oder zu verweigern,
	\item und dem \textit{Multi Channel Analyser}, der die Signalstärken bei den verschiedenen Kanälen darstellt.
\end{itemize}
%%%%%%%%%%%%%%%%%%%%%%%%%%%%%%%%%%%%%%%%%%%
%%%%%%%%%%%%%%%%%%%%%%%%%%%%%%%%%%%%%%%%%%%
\newpage
\section{Durchführung}
\label{sec:Durchführung und Resultate}
\subsection{Energiegesamtspektra; Kalibration Kanallage-Energie}
Um die Kanallagen als Energiewerte in keV zu interpretieren, muss eine zugehörige Kalibration durchgeführt werden. Wir beruhen uns auf der Tatsache, dass die Kanallagen des TSCA linear proportional zur Energie sind \cite{manualold}:
\begin{align}
	E_{\gamma} &= a + b\cdot K,
	\label{eq:calibrate}
\end{align}
mit den zu bestimmenden Parametern $a,b$ und der Kanallage $K$. \\
Wenn in einem Spektrum zwei Werte auf bestimmten Kanallagen sitzen und deren Energiewerte bekannt sind (wie z.B. die gaußschen Peaks der Gammaquantenenergien), kann man die Parameter $a$ und $b$ bestimmen:
\begin{align}
	\Delta E_{\gamma} &= b \Delta K , \\
	b &= \frac{\Delta E_{\gamma}}{\Delta K},
\end{align}
Die Peaks der Gammaquantenenergien können neben dem visuellen Ablesen durch numerische Fits genauer bestimmt werden. Für die multimodale Verteilung der Gesamtenergiespektren verwenden wir die Pythonumgebung \texttt{emcee} \cite{Foreman_Mackey_2013}, die den numerischen Fit mittels Markov-Chain-Monte-Carlo Sampling durchführt. Der Algorithmus ist dabei die sogenannte autocorrelation time Methode \cite{Foreman_Mackey_2013}. Die vertikalen Linien in Abbildung \ref{fig:so1initial} sind visuell abgelesene Schätzungen, die für das Angeben der Priors hilfreich sind.\\
Zur Veranschaulichung werden die Plots von einem Beispiel (Natrium Detektor 1) dargestellt. Die restlichen Plots sind im Anhang zu finden.
\begin{figure}[H]
		\centering
		\includegraphics[scale=.8]{"plots/Energiegesamtspektrum von Natrium aus Detektor 1".pdf}
		\caption{Gesamtenergiespektrum von Natrium, Detektor 1.
		Die zwei Peaks sind zunächst abgeschätzt.}
		\label{fig:so1initial}
\end{figure}
Für den numerischen Fit mittels \texttt{emcee} müssen wir einen Prior angeben; eine Modellfunktion, an der sich der numerische Fit zu richten hat. Wir konstruieren die Modellfunktion als Summe dreier Verteilungsfunktionen: eine Funktion für die Hintergrundstrahlung (lognormal bzw. exponentielle Verteilung) und zwei Normalverteilungsfunktionen für die Peaks der Gammaquantenenergien:
\begin{align}
	M(\vec{\theta}, x) &= a_1 \cdot P_{ln}((\mu_1,\sigma_1),x) + a_2 \cdot P_{N}((\mu_2,\sigma_2), x) + a_3 \cdot P_{N}((\mu_3,\sigma_3), x) \\
	P_{ln}((\mu,\sigma), x) &= \frac{1}{x \cdot \sigma \sqrt{2\pi}} \exp \left[ -\frac{1}{2} \left ( \frac{\ln x - \mu}{\sigma} \right )^2  \right] \\
    P_{gauss}((\mu,\sigma), x) &= \frac{1}{\sigma \sqrt{2\pi}} \exp \left[ -\frac{1}{2} \left( \frac{x-\mu}{\sigma} \right)^2 \right ] \\
    \vec{\theta} &= (\mu_1, \mu_2, \mu_3, \sigma_1, \sigma_2, \sigma_3, a_1, a_2, a_3)^T
\end{align}
mit dem Hyperparametervektor $\vec\theta$, dessen Werte der numerische Fit bestimmt. Die Werte des Hyperparameters sind im Anhang bei den Cornerplots zu entnehmen. 
\begin{figure}[H]
	\centering
	\includegraphics[scale=.8]{"plots/MCMC Natrium Detektor 1 Chains".pdf}
	\caption{MCMC Durchlauf Natrium, Detektor 1. Rot dargestellt sind die Markov-Ketten: Modellfunktionen mit variierendem Hyperparameter.}
\end{figure}

\begin{figure}[H]
	\centering
	\includegraphics[scale=.8]{"plots/MCMC Natrium Detektor 1 Fit".pdf}
	\caption{MCMC highest likelihood Natrium, Detektor 1. Das Modell mit der höchsten Likelihood. Grau markiert ist der 1$\sigma$-Bereich.}
\end{figure}

Für die Unsicherheit der gaußschen Mittelwerte verwenden wir entsprechend die gaußschen Standardabweichungen:
\begin{table}[H]
    \centering
    \begin{tabular}{llr|r|r|r}
	\toprule
	 &  & Kanallage Natrium $K,Na$ & $\sigma_{K,Na}$ & Kanallage Cobalt $K,Co$ & $\sigma_{K,Co}$ \\
	 & Peak &  &  &  &  \\
	\midrule
	\multirow[t]{2}{*}{Theoretischer Wert E} & p1 & 511.0000 & 0.0000 & 1173.2000 & 0.0000 \\
	 & p2 & 1275.0000 & 0.0000 & 1332.5000 & 0.0000 \\
	\cline{1-6}
	\multirow[t]{2}{*}{Detektor 1} & p1 & 482.4166 & 21.9667 & 1144.5461 & 43.5131 \\
	 & p2 & 1257.0498 & 32.8904 & 1304.0668 & 34.7581 \\
	\cline{1-6}
	\multirow[t]{2}{*}{Detektor 2} & p1 & 201.3974 & 28.6677 & 702.3187 & 50.9998 \\
	 & p2 & 783.2746 & 49.9971 & 828.7921 & 46.7660 \\
	\cline{1-6}
	\bottomrule
	\end{tabular}
	\caption{Erwartungswerte und Ungenauigkeiten der gaußschen Fits, entnommen von den Hyperparameters der MCMC-Fits.}
	\label{tab:mcmcerror}
\end{table}
\newpage
Die Kalibrationsparameter $a$ und $b$ haben eine Unsicherheit, die durch die gaußsche \\
Fehlerfortpflanzung bestimmt werden können: 
\begin{align}
    b                &= \frac{\Delta E_\gamma}{\Delta K} \\
    \Delta b         &=  \sqrt{ \left ( \frac{\partial b}{\partial (\Delta K)}\Delta(\Delta K) \right )^2 } \\
                     &= \frac{\Delta E_{\gamma}}{(\Delta K)^2}\Delta (\Delta K) \\
    \Delta(\Delta K) &= \sqrt{ \left( \frac{\partial(\Delta K)}{\partial K_1} \sigma_1 \right)^2 + 
        \left( \frac{\partial(\Delta K)}{\partial K_2} \sigma_2 \right)^2} \\
                     &= \sqrt{\sigma_1^2 + \sigma_2^2}\\
    a                &= E_{\gamma} - b \cdot K \\
    \Delta a         &= \sqrt{ \left( \frac{\partial a}{\partial b} \Delta b \right)^2 + 
        \left( \frac{\partial a}{\partial K} \sigma \right)^2  } \\
                     &= \sqrt{ \left( K \cdot \Delta b \right)^2 + \left( b \cdot \sigma \right)^2 }
\end{align}
Die Fehlerberechnung von $a$ entspricht dem formellen Prozess der gaußschen Fehlerfortpflanzung, weist jedoch hohe Werte auf, da sie - vor allem bei Cobalt - mit hohen Kanallagewerten $K$ skaliert, siehe dazu die Fehlerwerte in Tabelle \ref{tab:calibration}. Um qualitativ die Genauigkeit des Parameters $a$ zu untersuchen, führen wir eine weitere Kalibration mit Cobalt durch, siehe Abbildung \ref{fig:calibratecurve}. Wir erkennen dabei, dass Detektor 1 bei beiden Quellen eine beinahe gleiche Kalibration entsteht, während wir bei Detektor 2 eine ebenfalls guten Überlapp zwischen rot und blau, wenn auch mit einer stärkeren Diskrepanz sehen. Dieser visuelle Vergleich schließt die hohe Ungenauigkeit von $a$ nicht aus, regt jedoch dazu an, andere Alternativen zur gaußschen Fehlerfortpflanzung in Erwägung zu ziehen, denn der visuelle Abgleich bei Detektor 1 deutet darauf hin, dass der Parameter $a$ zu hoher Wahrscheinlichkeit den richtigen Wert annimmt. \\
Gemäß der Fehlerrechnung ergibt sich bei beiden Detektoren jeweils ein Energiewert von
\begin{align}
	E_{\gamma, D1} [\text{in $keV$}]&= (0.9862 \pm 0.0504)\cdot K + (35.2054 \pm 23.5506) [\text{in $keV$}]\\
	E_{\gamma, D2} [\text{in $keV$}]&= (1.3130 \pm 0.0915)\cdot K + (246.5669 \pm 41.9077) [\text{in $keV$}].
\end{align}
Für die Bestimmung der Fehlerbalken für die Natriumeichung, die blau in Abbildung \ref{fig:calibratecurve} eingezeichnet sind, wird die gaußsche Fehlerfortpflanzung für die konkret berechneten Energien bei den Kanallagenpeaks durchgeführt:
\begin{align}
	\Delta E_{\gamma,i} &= \sqrt{(b_i \cdot \sigma_{K,Na,i})^2 + (K_{Na,i} \cdot \Delta b_i)^2 + (\Delta a_i)^2},
\end{align}
wobei der Index $i = 1,2$ den Detektor angibt.

\begin{table}[H]
	\centering
	\begin{tabular}{ll|rr|rr}
	\toprule
	 &  & Natrium & Unsicherheit $\Delta_k$ & Cobalt & Unsicherheit $\Delta_k$ \\
	Detektor & Kalibrationsparameter $k$ &  &  &  &  \\
	\midrule
	\multirow[t]{2}{*}{D1} & a & 35.2054 & 32.5506 & 30.2370 & 401.3858 \\
	 & b & 0.9863 & 0.0504 & 0.9986 & 0.3486 \\
	\cline{1-6}
	\multirow[t]{2}{*}{D2} & a & 246.5669 & 41.9077 & 288.5921 & 488.2277 \\
	 & b & 1.3130 & 0.0915 & 1.2596 & 0.6891 \\
	\cline{1-6}
	\bottomrule
	\end{tabular}
\caption{Parameter der Kalibration}
\label{tab:calibration}
\end{table}

%\begin{table}[H]
%	\centering
%	\begin{tabular}{llrr}
%	\toprule
%	&  & Na & Co \\
%	Detektor & Kalibrationsparameter &  &  \\
%	\midrule
%	\multirow[t]{2}{*}{D1} & a & 35.2252 & 35.2281 \\
%	 & b & 0.9865 & 0.9944 \\
%	\cline{1-4}
%	\multirow[t]{2}{*}{D2} & a & 246.8638 & 293.6654 \\
%	 & b & 1.3113 & 1.2520 \\
%	\cline{1-4}
%	\bottomrule
%	\end{tabular}
%
%\end{table}

\begin{figure}[H]
	\centering
	\includegraphics[scale=.9]{plots/Eichgeraden.pdf}
	\caption{Eichgerade. Es wurde sowohl eine Natrium, als auch eine Cobalt Eichung durchgeführt. Zweck ist der qualitative Vergleich der Detektoren und der Kalibrationsungenauigkeit.}
	\label{fig:calibratecurve}
\end{figure}
Wir verwenden die Natriumeichung, um die Energie für eine gegebene Kanallage gemäß \ref{eq:calibrate} zu bestimmen. Ein kalibriertes Gesamtspektrum ist in Abbildung \ref{fig:calibratespectrum} zu sehen, für die restlichen Spektren siehe Anhang \ref{fig:calibratedsodium1} bis \ref{fig:calibratedcobalt2}. 
\begin{figure}[H]
	\centering
	\includegraphics[scale=.9]{"plots/Kalibriertes Energiegesamtspektrum von Cobalt aus Detektor 1".pdf}
	\caption{Kalibriertes Energiegesamtspektrum von Cobalt aus Detektor 1. Bei allen vier Plots wird die Natriumeichung verwendet.}
	\label{fig:calibratespectrum}
\end{figure}
\newpage
\subsection{Time-delay Spektra, Kalibration Kanallage-Zeit}
Wir führen eine zum vorherigen Abschnitt analoge Kalibration durch. Bei dieser Messung folgen die Daten einer unimodalen Verteilung, weshalb wir für den numerischen Fit die Umgebung \texttt{scipy.optimize} verwenden. Der Vorteil hierbei ist, dass der Fitprozess signifikant schneller durchgeführt werden kann. Als Vergleich wird noch eine "per Hand" berechnete, analytische Gaußfunktion zum Verleich dargestellt. Die parametrisierten Fits sind in der Abbildung \ref{fig:timedelay} und Abbildung \ref{fig:timedelay96} zu sehen. 
\begin{figure}[H]
	\centering
	\includegraphics[scale=.8]{"plots/Vergleich gaußscher Fits bei 0ns Delay".pdf}
	\caption{Numerischer Fit für 0ns}
	\label{fig:timedelay}
\end{figure}
\begin{figure}[H]
	\centering
	\includegraphics[scale=.8]{"plots/Vergleich gaußscher Fits bei 96ns Delay".pdf}
	\caption{Numerischer Fit für 96ns}
	\label{fig:timedelay96}
\end{figure}
Zum Abgleich der Kalibration verwenden wir hier die Zeitdifferenz von $96$ns und $0$ns und die Differenz in den Peaks der Kanallagen:
\begin{align}
	\Delta t &= 96ns = b \cdot \Delta K. 
\end{align}
Für die Fehlerfortpflanzung verwenden wir die vom numerischen Fit gegebenen Standardabweichungen $\sigma_{0}=100.3792$ und $\sigma_{96}= 100.7407$:
\begin{align}
    \Delta b &= \frac{\Delta t}{(\Delta K)^2} \Delta(\Delta K) \\
    \Delta (\Delta K) &= \sqrt{ \sigma_1^2 + \sigma_2^2 }.
\end{align}
Die Eichgerade ist in Abbildung \ref{fig:timecalibrate} zu sehen. Wir kommen auf eine Kalibration von
\begin{align}
	t (\text{in $ns$})&= (0.1428 \pm 0.0302) \cdot K \\
	\Delta t_{i} &= \sqrt{ (K_i \Delta b)^2 + (b \sigma_i)^2 }
\end{align}
mit $i $= 0$ns$, 96$ns$.
\begin{figure}[H]
	\centering
	\includegraphics[scale=.8]{"plots/Eichgerade Zeit".pdf}
	\caption{Eichgerade Kanallage-Zeit}
	\label{fig:timecalibrate}
\end{figure}
Mit der Zeiteichung können wir die Zeitauflösung bestimmen. Sie ergibt sich aus der Bestimmung der Breite der Gaußen Verteilungen bei halber Höhe (FWHM: Full Width at Half Maximum). Dazu kann entweder der x-Wert bei halber Höhe bestimmt werden, oder die Standardabweichung verwendet werden, um die Breite direkt zu bestimmen: $\text{FWHM} =2 \sqrt{2\ln 2}\sigma$ \cite{FWHM}. Ersteres kann numerisch mit \texttt{numpy.interp} durchgeführt werden, in dessen Dokumentation keine Ungenauigkeitbestimmung gegeben ist. Die Ungenauigkeit bestimmen wir daher nur im Bezug auf die Ungenauigkeit des Kalibrationsparameters:
\begin{align}
	\Delta \text{FWHM} &= \sqrt{ (x_2 - x_1)^2\Delta^2 b }.
\end{align}
Als Endwerte bekommen wir
\begin{table}[H]
	\centering
	\begin{tabular}{lll}
		\toprule
			Zeitdelay & FWHM         & Ungenauigkeit \\
		\midrule
			0$ns$     & 32.80362$ns$ & 6.93402$ns$ \\
		\cline{1-3}
			96$ns$    & 32.96602$ns$ & 6.96835$ns$ \\
		\cline{1-3}
		\bottomrule
	\end{tabular}
	\caption{FWHM Werte mit Ungenauigkeiten}
	\label{tab:fwhm}
\end{table}

\newpage
\subsection{Koinzidenzmessungen}
\noindent \textbf{$e^+e^-$ Annihilation (511keV - 511keV)}\\
	Es wird die Vernichtungsstrahlung von den Positronen aus der Natriumquelle bei verschiedenen kleinen Winkelauslenkungen betrachtet:

\begin{table}[H]
    \centering
    \begin{tabular}{llll}
    \hline
        Auslenkung & Gesamtzahl $Z$ & Gesamtmesszeit $T$ & Zählraten $Q$\\ \hline
        0cm & 24 & 123.136s & 0.1949064449064449 $s^{-1}$ \\ \hline
        -1cm & 59 & 123.155s & 0.47907108927773945 $s^{-1}$ \\ \hline
        -2cm & 84 & 129.882s & 0.6467408878828474 $s^{-1}$ \\ \hline
        -3cm & 48 & 122.609s & 0.3914883899224364 $s^{-1}$ \\ \hline
    \end{tabular}
	\caption{Zählraten bei verschiedenen Auslenkungen für die Vernichtungsstrahlung}
	\label{tab:coincidenceannihilation}
\end{table}

Eine grafische Darstellung ist in Abbildung \ref{fig:coincidenceannihilation} zu sehen. Für die Messungenauigkeit in der Auslenkung werden 0.2$cm$ angenommen, während die Ungenauigkeit der Gesamtzahl sich in die Ungenauigkeit der Zählrate fortpflanzt. Die Gesamtzahl ist eine diskrete, poissonverteilte Größe, deren Standardabweichung die Wurzel des Erwartungswertes $\Delta Z = \sqrt{Z}$ ist. So gilt dann für die Ungenauigkeit der Zählrate $Q$:
\begin{align}
	\Delta Q &= \sqrt{ \left( \frac{1}{T}\sqrt{Z}  \right)^2  }
\end{align}

\begin{figure}[H]
	\centering
	\includegraphics[scale=0.8]{"plots/Koinzidenzplot Vernichtungsstrahlung".pdf}
	\caption{grafische Darstellung der $\gamma - \gamma -$Koinzidenz für die Vernichtungsstrahlung}
	\label{fig:coincidenceannihilation}
\end{figure}
Wir erkennen, dass - anders als von der Theorie erwartet - die maximale Signalstärke nicht bei 180°-Auslenkung der beiden Detektoren liegt, sondern etwas verschoben. Aus der Theorie erwarten wir, dass das Positron bei der Annihilation thermische/nichtrelativistische Geschwindigkeiten besitzt, und daher die Impulserhaltung mit null Anfangsimpuls gilt. Ein Grund für die von der Theorie abweichenden Impulsrichtungen ist der ungenaue Versuchsaufbau, der zu systematischen Fehlern führen kann. \vspace{10pt}\\
\textbf{$e^+e^- $ Annihilation und $\gamma$-Zerfall (511keV - 1275keV)}\\
Es wird erwartet, dass die gemessenen Zählraten kleiner werden als beim vorherigen Versuch. Dies konnte in unserem Fall nicht bestätigt werden. Eine grafische Darstellung ist in Abbildung \ref{fig:coincidenceannihilgamma} zu sehen. Wir erwarten aus der Theorie keine Koinzidenz zwischen den Gammaquanten der Vernichtung und des Gammazerfalls, da die zwei Prozesse unabhängig voneinander ablaufen. Wir erwarten also keine Zeitgleichheit aus einem kausalen Zusammenhang. Aufgrund der endlichen Auflösung ist jedoch zu erwarten, dass das Zeitfenster zwischen erstem und zweiten Gammaquant sehr kurz ist, und zu zufälligen Koinzidenzen führen kann. \\
Würde eine Koinzidenz existieren, würden wir in der Abblidung eine Winkelabhängigkeit gemäß der Winkelkorrelationsfunktion erkennen. \\
Für die Messungenauigkeit im Auslenkwinkel werden 2° angenommen, während wir wiederum eine poissonverteilte Ungenauigkeit $\sqrt{Z}$ für die Gesamtzahl $Z$ annehmen. 

\begin{figure}[H]
	\centering
	\includegraphics[scale=0.8]{"plots/Koinzidenzplot Vernichtungsstrahlung Gammazerfall".pdf}
	\caption{grafische Darstellung der $\gamma - \gamma -$Koinzidenzmessung für die Vernichtungsstrahlung und den Gammazerfall}
	\label{fig:coincidenceannihilgamma}
\end{figure}

\textbf{Kaskadenzerfall Cobalt}\\
Wir bestimmen die Quellstärke/Zerfallsrate $Q$ für $W(55)\approx 1$ gemäß Gl. \ref{eq:qvalue}. Es ist hierbei wichtig anzumerken, dass hier die Größen $Z$ nicht die Gesamtzahl, sondern Zählraten sind, sodass die folgende Ungenauigkeit für $Q$ gilt:
\begin{align}
	\Delta Q &= \sqrt{ \left(\frac{Z_2}{2 \cdot Z_{Co}}\frac{\sqrt{Z_1 \cdot T_1}}{T_1} \right)^2 + \left(\frac{Z_1}{2 \cdot Z_{Co}}\frac{\sqrt{Z_2 \cdot T_2}}{T_2} \right)^2 + \left(\frac{Z_1 \cdot Z_2}{2 \cdot Z_{Co}^2}\frac{\sqrt{Z_{Co} \cdot T_{Co}}}{T_{Co}} \right)^2  },
\end{align}
bei der die Wurzelterme die Ungenauigkeit der Gesamtzahl darstellen sollen und daher die Werte $Z$ mit ihrer jeweiligen Messzeit multipliziert werden. Wir erhalten für die Zerfallsrate
\begin{align}
	Q &= (23107.6514 \pm 807.2960) \text{Bq}.
\end{align}
Für die Bestimmung des Winkelkorrelationsparameters $W(\vartheta)$ für die Winkel 10°, 40° und 80° stellen wir die Gl. \ref{eq:qvalue} entsprechend um. Für die Fehlerrechnung gilt:
\begin{align}
	\Delta W &= \sqrt{ \left( \frac{2 \cdot Z_K}{Z_1 \cdot Z_2} \Delta Q  \right)^2 + \left( \frac{2 \cdot Q}{Z_1 \cdot Z_2} \Delta Z_K  \right)^2 + \left( \frac{2 \cdot Z_K \cdot Q}{Z_1^2 \cdot Z_2} \Delta Z_1  \right)^2 + \left( \frac{2 \cdot Z_K \cdot Q}{Z_1 \cdot Z_2^2} \Delta Z_2  \right)^2 }
\end{align}

Tabelle \ref{tab:qtable} enthält die Werte der Messparameter $Z_1, Z_2, Z_K, Q$ und $W(\vartheta)$ mitsamt ihrere jeweiligen Ungenauigkeiten. Eine graphische Darstellung der Winkelkorrelation im Vergleich zu ihrem theoretischen Wert ist in Abbildung \ref{fig:coincidencecobalt} zu finden.
\begin{table}[H]
\centering
\begin{tabular}{lrrrrrrrr}
\toprule
 & $Z_1 [s^{-1}]$ & $\Delta_{Z1} [s^{-1}]$ & $Z_2 [s^{-1}]$ & $\Delta_{Z2} [s^{-1}]$ & $Z_K [s^{-1}]$ & $\Delta_{Zk} [s^{-1}]$ & $W(\vartheta)$ & $\Delta_W$ \\
Winkel &  &  &  &  &  &  &  &  \\
\midrule
10 & 304.9289 & 0.8505 & 293.4173 & 0.8343 & 2.2582 & 0.0732 & 1.1664 & 0.0558 \\
40 & 310.8005 & 0.8541 & 293.2085 & 0.8296 & 2.0467 & 0.0693 & 1.0380 & 0.0507 \\
55 & 309.4117 & 0.8564 & 293.8306 & 0.8345 & 1.9672 & 0.0683 & 1.0000 & 0.0494 \\
80 & 311.8168 & 0.8570 & 291.7363 & 0.8290 & 1.9833 & 0.0683 & 1.0076 & 0.0496 \\
\bottomrule
\end{tabular}
\caption{Messparameter des Kaskadenzerfalls}
\label{tab:qtable}
\end{table}

\begin{figure}[H]
	\centering
	\includegraphics[scale=0.8]{"plots/Winkelkorrelationen Vergleich".pdf}
	\caption{Vergleich der Winkelkorrelationsfunktion von gemessenen Daten und von der Theorie}
	\label{fig:coincidencecobalt}
\end{figure}

Abgesehen von den Fehlern sehen wir, dass die Daten der theoretischen Kurve nur auf sehr grober Weise folgen. Hinzu kommt, dass die Ungenauigkeit der Werte bei 10° und 80° die theoretischen Werte nicht enthält. Eine Koinzidenz der Gammaquanten aus dem Kaskadenzerfall ist zwar aufgrund ihres kausalen Zusammenhangs zu erwarten, kann aber von unserem Versuch nicht mit passenden Daten bekräftigt werden. 

%%%%%%%%%%%%%%%%%%%%%%%%%%%%%%%%%%%%%%%%%%%
%%%%%%%%%%%%%%%%%%%%%%%%%%%%%%%%%%%%%%%%%%%



\section{Diskussion}
\label{sec:Diskussion}
\textbf{Zu den kalibrierten Energiespektren}\\
Mit der Natriumeichung erreichen die Maxima der Cobaltspektren die theoretischen Werte von 1173.2$keV$ und 1332.5$keV$. Aus dem Vergleich der Natriumeichung mit der Cobalteichung sehen wir, dass sich die Kalibrationen, abgesehen von der leichten Diskrepanz bei Detektor 2, gut überlappen. Wir bestätigen somit die Qualität der Kalibration, die in Eichlabors wie die Physikalisch-Technische Bundesanstalt Braunschweig angewendet wird \cite{manual1}.\vspace{10pt}\\
\textbf{Technische Einzelheiten, Datenanalyse}\\
Eine Kategorie des Fehlers wurde in unserer Durchführung vernachlässigt, und zwar sind es die Fehler vom numerischen Fit mittels MCMC, die im Cornerplot \ref{fig:cornerplot} zu erkennen sind. Wir haben uns dazu entschieden, diese Fehler zunächst auszulassen, da der numerische Fit bereits bei jeder Programmausführung schwankt und der Fehler wenig aussagekräftig ist. Es muss also zunächst sichergestellt werden, dass der numerische Fit reproduzierbare Werte liefert, was durch ein Erhöhen der Iteration (im Moment 500) erreicht werden kann, jedoch würde dies die Ausführungszeit deutlich erhöhen; ein MCMC-Durchlauf dauert bei einem Prozessor mit 6 Kernen mit jeweils 2 Threads etwa eine Minute, bei 2 Kernen mit jeweils 2 Threads bereits fünf Minuten. Eine Lösung dazu wäre es, den Vorgang zu parallelisieren (\texttt{number of parallel jobs = }Anzahl an Kernen). Eine weitere Möglichkeit besteht darin, den Wertebereich von gut bekannten Parametern so klein wie möglich zu halten, was bedeutet, dass Parameterwerte über dem gegebenen Toleranzbeich hinaus gar nicht erst von den MCMC-Ketten akzeptiert werden. Im Programm wird dies durch die \texttt{lnprior}-Funktion ermöglicht. Dazu ist aber entweder ein gutes Vorwissen über die Parameterwerte oder viel Herumprobieren mit dem numerischen Fit erforerlich.\\
Ziel des numerischen Fits ist die Bestimmung der Maxima und dessen gaußischer Standardabweichung gewesen. Wir haben uns darauf beschränkt, den Highest-Likelihood Fit visuell zu kontrollieren, ob er mit den Datenpunkten übereinstimmt. Auch dies ist vor allem bei Daten mit mehreren Hintergründen nicht gewährleistet; Für die beiden Natriumspektren hat die Modellierung mit drei Funktionen (lognormaler Hintergrund, gaußscher Peak 511$keV$, gaußscher Peak 1275$keV$) gut funktioniert. Für Cobalt jedoch sehen wir (vor allem bei Detekor 1, siehe Abbildung \ref{fig:mcmcfitco1}), dass drei Modellfunktionen nicht ausreichen. Wir vermuten, dass es mehrere Mechanismen für den Hintergrund gegeben hat: Neben der Compton-Rückstreuung können beispielsweise Photonen im Röntgenbereich aus der photoelektrischen Absorption von einem Detektor zum anderen gestreut worden sein. Weitere, jedoch unwahrscheinliche Möglichkeiten sind Aktivitäten von anderen zerfallenden Quellen.\\
Eine weitere, bisher ununtersuchte jedoch vielversprechende Methode des numerischen Fits für multimodlae Verteilungen ist die \textit{KDE,} die kernel density estimate.\vspace{10pt}\\
\textbf{Zur Zeitauflösung}\\
Gemäß Gilmore et.\textit{al} gilt für dem FWHM von Germaniumdetektoren der Wertebereich 3-10ns (\cite{gilmore2008practical} S. 52), eine deutlich bessere Zeitauflösung. Entsprechend dieser Literatur wäre ein Abgleich mit Herstellerangaben nützlich, um den von uns berecheten Wert bezüglich der Durchführungsqualität und der Detektroqualität zu interpretieren. Eine Angabe zu unserem Equipment ist in \cite{ortec} zu finden, jedoch haben wir bisher keine Angaben zur Zeitauflösung gefunden.\vspace{10pt}\\
\textbf{Zur Koinzidenzmessung}\\
Es besteht die Möglichkeit, die Quellstärke $Q$ von Cobalt-60 nachzukontrollieren, wenn die Masse der im Versuch verwendeten Probe bekannt ist.\\
Wie in der Durchführung besprochen, sind die Daten der Winkelkorrelation schlecht als Theoriebestätigung zu akzeptieren. Es liegt die Vermutung nahe, dass noch ein Schritt in der Durchführung fehlt, und zwar von den Zählraten den Hintergrundanteil aus den lognormalen Fits abzuziehen; dies würde die Zerfallsrate erniedrigen und möglicherweise die Theorie besser bestätigen. Die Gültigkeit dieses Ansatzes muss jedoch zunächst bestätigt werden, denn bisher sind noch keine theoretischen Modelle für die Hintergrundmechanismen gut bekannt außer der Comptonstreuung. Zu den Mechanismen zählen die Photonen aus der photolelektrischen Absorption und Hintergrundaktivitäten von unbekannten, unerwünschten Quellen. Gelingt es uns, gute theoretische Modelle für die Hintergrundmechanismen zu finden, wäre dann der Ansatz die lognormale Funktion bei den kalibrierten Maxima auszuwerten und den Wert als Hintergrundprozess von den Zählraten der jeweiligen Detektoren abzuziehen. 

%%%%%%%%%%%%%%%%%%%%%%%%%%%%%%%%%%%%%%%%%%%
%%%%%%%%%%%%%%%%%%%%%%%%%%%%%%%%%%%%%%%%%%%




\section{Fazit}
\label{sec:Fazit}
Mit diesem Versuch bestätigen wir die Genauigkeit der Natriumeichung. Für das Modellieren multimodaler Verteilungen untersuchen wir die Markov-Chain-Monte-Carlo Methode für das numerische Fit an Daten und finden, dass bei guter Modellierung präzise Fits für den Gesamtspektrum der Daten erstellt werden können. \\
Wir lernen, dass es wichtig ist, die Zeitauflösung gut zu verstehen, da die Qualität der Koinzidenzmessung von dieser Größe zu hohem Maß abhängt. Für genauere Auswertungen schlagen wir vor, dass über die Modellierung der Hintergrundaktivität nachgedacht werden soll. 

\newpage

%%%%%%%%%%%%%%%%%%%%%%%%%%%%%%%%%%%%%%%%%%%
%%%%%%%%%%%%%%%%%%%%%%%%%%%%%%%%%%%%%%%%%%%
\interlinepenalty=10000
\section{Quellen}
\setlength{\bibsep}{6pt}
\bibliography{report.bib}

%%%%%%%%%%%%%%%%%%%%%%%%%%%%%%%%%%%%%%%%%%%
%%%%%%%%%%%%%%%%%%%%%%%%%%%%%%%%%%%%%%%%%%%


\section{Anhang}
\label{sec:Anhang}
\begin{figure}[ht]
	\label{fig:initialplotsSo}
	\begin{subfigure}[c]{0.8\textwidth}
		\includegraphics[width=\textwidth]{"plots/Energiegesamtspektrum von Natrium aus Detektor 1".pdf}
		\subcaption{Natrium, Detektor 1}
	\end{subfigure}

	\begin{subfigure}[c]{0.8\textwidth}
		\includegraphics[width=\textwidth]{"plots/Energiegesamtspektrum von Natrium aus Detektor 2".pdf}
		\subcaption{Natrium, Detektor 2}
	\end{subfigure}
\end{figure}

\begin{figure}[ht]
	\label{fig:initialplotsCo}
	\begin{subfigure}[c]{0.8\textwidth}
		\includegraphics[width=\textwidth]{"plots/Energiegesamtspektrum von Cobalt aus Detektor 1".pdf}
		\subcaption{Cobalt, Detektor 1}
	\end{subfigure}

	\begin{subfigure}[c]{0.8\textwidth}
		\includegraphics[width=\textwidth]{"plots/Energiegesamtspektrum von Cobalt aus Detektor 2".pdf}
		\subcaption{Cobalt, Detektor 2}
	\end{subfigure}

\end{figure}
\begin{figure}		
	\includegraphics[scale=.3]{"plots/MCMC Natrium Detektor 1 Hyperparameters".pdf}
	\caption{MCMC Hyperparameter corner plot, Natrium, Detektor 1.}
	\label{fig:cornerplot}
\end{figure}

\begin{figure}
	\includegraphics[scale=.8]{"plots/MCMC Natrium Detektor 2 Fit".pdf}
	\caption{Highest Likelihood Fit Natrium, Detektor 2}
	\label{fig:mcmcfitna2}
\end{figure}

\begin{figure}
	\includegraphics[scale=.8]{"plots/MCMC Cobalt Detektor 1 Fit".pdf}
	\caption{Highest Likelihood Fit Cobalt, Detektor 1}
	\label{fig:mcmcfitco1}
\end{figure}

\begin{figure}
		\includegraphics[scale=.8]{"plots/MCMC Cobalt Detektor 2 Fit".pdf}
	\caption{Highest Likelihood Fit Cobalt, Detektor 2}
	\label{fig:mcmcfitco2}
\end{figure}

\begin{figure}		
	\includegraphics[scale=.8]{"plots/Kalibriertes Energiegesamtspektrum von Natrium aus Detektor 1".pdf}
	\caption{kalibriertes Spektrum Natrium, Detektor 1.}
	\label{fig:calibratedsodium1}
\end{figure}



\begin{figure}		
	\includegraphics[scale=.8]{"plots/Kalibriertes Energiegesamtspektrum von Natrium aus Detektor 2".pdf}
	\caption{kalibriertes Spektrum Natrium, Detektor 2.}
\end{figure}
\begin{figure}		
	\includegraphics[scale=.8]{"plots/Kalibriertes Energiegesamtspektrum von Cobalt aus Detektor 2".pdf}
	\caption{kalibriertes Spektrum Cobalt, Detektor 2.}
	\label{fig:calibratedcobalt2}
\end{figure}

\end{document}
